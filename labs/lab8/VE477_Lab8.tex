\documentclass[12pt, a4paper]{article}
\usepackage[UTF8]{ctex}
\usepackage{enumerate}
\usepackage{amsmath}
\usepackage{amssymb}
\usepackage{blkarray}
\usepackage{geometry}
\usepackage{float}
\usepackage{graphicx}
\usepackage[linesnumbered, ruled]{algorithm2e}
\usepackage{listings}
\usepackage{xcolor}


\geometry{left = 2.0cm, right = 2.0cm}


\title{
    \begin{figure}[H]
        \centering
        \includegraphics{AAA.png}
    \end{figure}
    VE477 Introduction to Algorithms\\ 
    Lab 8}
\author{Taoyue Xia, 518370910087}
\date{2021/12/07}

\begin{document}
\maketitle

\newpage
\section*{Ex. 1 --- General questions}
\begin{enumerate}
    \item Linear programming is a method to achieve the best outcome (such as maximum profit or lowest cost) in a mathematical model, 
          whose requirements are represented by linear constraints and a linear objective function.
    \item Optimizations, such as the Hitchcock Transport Problem, and problems in microeconomics.
    \item \textbf{Standard Form}, consists the following three parts:
          \begin{itemize}
              \item A linear function to be maximized, e.g., $f(x_1, x_2) = c_1x_1 + c_2x_2$.
              \item Problem Constraints like

                    \[
                        \begin{cases}
                            a_{11}x_1 + a_{12}x_2 &\leq b_1\\
                            a_{21}x_1 + a_{22}x_2 &\leq b_2\\
                            a_{31}x_1 + a_{32}x_2 &\leq b_3
                        \end{cases}    
                    \]
              \item Non-negative variables, e.g., $x_1 \leq 0,\ x_2\leq 0$.

          \end{itemize}

          \textbf{Slack Form}, written in a matrix form to, for example, maximize $z$ like
          \[\begin{bmatrix} 1 & -c^T & 0\\ 0 & A & I \end{bmatrix} \begin{bmatrix} z\\ x\\ s \end{bmatrix} = \begin{bmatrix}
              0\\ b \end{bmatrix}\]
          where $s$ are slack variables, $x$ are decision variables, and $z$ is the variable to be maximized.

          \textbf{Standard Form} directly expresses the requirements of the optimization.

          \textbf{Slack Form} uses slack variables to compensate the inequality to become an equation, 
          because of which the equations can be written in matrix form to be calculated using linear algebra.
    \item Algorithms are 
          \begin{itemize}
              \item The simplex algorithm
              \item Criss-cross algorithm
              \item Ellipsoid algorithm
              \item $\dots$
          \end{itemize}

          The simplex method operates the requirements in the canonical form,
          \begin{itemize}
              \item Maximize $c^Tx$
              \item subject to $Ax \leq b$ and $x \geq 0$
          \end{itemize}
          where $c = \{c_1, \dots, c_n\}$ the coefficients of the objective function, 
          $x = \{x_1, \dots, x_n\}$ are the variables of the problem, 
          $A$ is a $p\times n$ matrix, which represents for $p$ constrains, 
          and $b = \{b_1, \dots, b_p\}$.

          This method is implemented similarly to the Slack Form, but some difference exists.

          We would focus on the problem to find a maximum. 
          After constructing the initial matrix, we should firstly decide on the pivot column, ranging from 2 to $n+1$. 
          Then after we choose column $i$, decide on the pivot row, by dividing $b_j$ by $A_{ji}$ as $d_j = b_j / A_{ji}$. 
          Choose the row with the minimum $d_j$. Then use Gaussian elimination to make all $m_{ki}(k \leq j)$ be 0. 
          Finally, the first row would give the maximum result.
    \item Duality is the principle that optimization problems may be viewed from either of two perspectives, 
          the primal problem or the dual problem.

          For example, if we have an optimization problem which tells us to maximize $c^Tx$ subject to $Ax \leq b$, $x \geq 0$, 
          the problem can be transformed into the corresponding asymmetric dual problem, 
          minimize $b^Ty$ subject to $A^Ty = c$, $y \geq 0$.
          This transformation is used in the simplex method.
\end{enumerate}

\section*{Ex. 2 --- Toy example for the simplex method}
\begin{enumerate}
    \item \begin{enumerate}[a)]
        \item In standard form, 
              
              The linear function is
              \[y = -2x_1 + 3x_2\]
              The constraints are
              \[
                \begin{cases}
                    x_1 + x_2 &= 7\\
                    x_1 - 2x_2 &\leq 4\\
                \end{cases} 
              \]
              Non-negative variables
              \[x_1 \geq 0\]
        \item In slack Form
        
              The matrix equation can be shown as
              \[
                \begin{bmatrix}
                    1 & 2 & -3 & 0 \\
                    0 & 1 & 1  & 0 \\
                    0 & 1 & -2 & 1
                \end{bmatrix}
                \begin{bmatrix}
                    y\\ x_1\\ x_2\\ x_3
                \end{bmatrix}
                =
                \begin{bmatrix}
                    0\\ 7\\ 4
                \end{bmatrix}
              \]

    \end{enumerate}
    \item First write the system of equations into matrix form, which is 
          \[
            \begin{bmatrix}
                1 & -3 & -1 & -2 & 0 & 0 & 0 & 0\\
                0 &  1 &  1 &  3 & 1 & 0 & 0 & 30\\
                0 &  2 &  2 &  5 & 0 & 1 & 0 & 24\\
                0 &  4 &  1 &  2 & 0 & 0 & 1 & 36
            \end{bmatrix}    
          \]
          Then choose column 2 to be the pivot column, and we calculate the division of row 2, 3, 4 at column 3,
           to get the values for $x_1$. For row 2, $30/1 = 30$. For row 3, $24/2 = 12$. For row 4, $36/4 = 9$. 
           So we choose row 4 to be the pivot row, and apply row operation, then the matrix looks like:
           \[
            \begin{bmatrix}
                4 &  0 & -1 & -2 & 0 & 0 & 3 & 108\\
                0 &  0 & 3 & 10 & 4 & 0 & -1 & 94\\
                0 &  0 & 3 & 8 & 0 & 2 & -1 & 12\\
                0 &  4 & 1 & 2 & 0 & 0 &  1 & 36
            \end{bmatrix}    
           \]
           The corresponding feasible solution is $x_1 = x_6 = 0$.
           
            Then we choose column 4 and row 3, it gives
            \[
            \begin{bmatrix}
                16 &  0 & -1 & 0 & 0 & 2 & 11 & 444\\
                0 &  0 & -3 & 0 & 16 & -10 & 1 & 316\\
                0 &  0 & 3 & 8 & 0 & 2 & -1 & 12\\
                0 &  16 & 1 & 0 & 0 & -2 &  5 & 132
            \end{bmatrix}
            \]
           Then we choose column 3 and row 3, it gives
           \[
            \begin{bmatrix}
                48 &  0 & 0 & 8 & 0 & 8 & 32 & 1344\\
                0 &  0 & 0 & 8 & 16 & -8 & 0 & 328\\
                0 &  0 & 3 & 8 & 0 & 2 & -1 & 12\\
                0 &  48 & 0 & -8 & 0 & -4 &  16 & 384
            \end{bmatrix}
            \]

           Then it reaches the end, since the coefficient of $x_i$ are all non-positive. So we can see that
           \[z = 28 - \frac{x_3}{6} - \frac{x_5}{6} - \frac{2x_6}{3} \]
           so we can conclude that z can reach the maximum 28.
\end{enumerate}


\end{document}